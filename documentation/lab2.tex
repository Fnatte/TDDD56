\documentclass[a4paper,12pt]{article}
\usepackage[utf8]{inputenc}
\usepackage[english]{babel}
\usepackage{graphicx}
\DeclareGraphicsExtensions{.pdf,.png,.jpg}
\usepackage{listings}
\usepackage{hyperref}
\lstset{
language=C,
basicstyle=\footnotesize
}
\begin{document}

\section{Lab 2}
\subsection{Before the lab session}

\begin{itemize}
\item \textit{Write an explaination on how CAS can be used to implement protection for concurrent use of data structures.}

  Compare And Swap (CAS) can be used to implement a lock free concurrent stack by, for example:

  A stack has a head pointer and back pointer. A list item has a next pointer and previous pointer. When we insert an item new\_item into the stack we read the head pointer into last\_item and do a CAS on list.head-$\rangle$next; if head-$\rangle$next == null, swap in new\_item. Next we do a CAS on head pointer; if list.head == last\_item, swap in new\_item.


\item \textit{Sketch a scenario featuring several threads raising the ABA problem.}

  See hand written notes.


\item \textit{Measure and plot the performance.}

\begin{figure}[h]
  \centering
  \includegraphics[width=0.9\textwidth]{global_timing_both.png}
  \caption{Graph laying it all bare.}
\end{figure}


\item \textit{Explain how CAS can implement a safe data structure sharing between several threads}

By making sure that no one has been changing the stack while we were operating on it we can, sort of, make sure that the CAS stack is a thread safe data structure. If someone has modified the stack, we try again, over and over until we succeed in modifying the stack without anyone else getting between our non-atomic operations.


\item \textit{Execute a multi-threaded implementation of the test_aba() unit test}

See how wonderfully it fails as we switch to the ABA branch.

\end{itemize}

\end{document}
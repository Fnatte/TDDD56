\documentclass[a4paper,12pt]{article}
\usepackage[utf8]{inputenc}
\usepackage[english]{babel}
\usepackage{graphicx}
\DeclareGraphicsExtensions{.pdf,.png,.jpg}
\usepackage{listings}
\usepackage{hyperref}
\lstset{
language=C,
basicstyle=\footnotesize
}
\begin{document}

\section{Lab 3}


\begin{itemize}
\item \textit{How many cores will simple.cu use, max, as written? How many SMs?}

  It will at maximum use 16 * 1 cudacores. It will run on one Streaming Multiprocessor (SM).

\item \textit{Is the calculated square root identical to what the CPU calculates?}

  It is not identical:

  \begin{lstlisting}
    0 0
    1 1
    1.4142136573791504 1.4142135623730951
    1.732050895690918  1.7320508075688772
    2 2
    2.2360682487487793 2.2360679774997898
    2.4494898319244385 2.4494897427831779
    2.6457514762878418 2.6457513110645907
    2.8284273147583008 2.8284271247461903
    3.0000002384185791 3
    3.1622776985168457 3.1622776601683795
    3.3166248798370361 3.3166247903553998
    3.4641017913818359 3.4641016151377544
    3.6055512428283691 3.6055512754639891
    3.7416574954986572 3.7416573867739413
    3.872983455657959  3.872983346207417
  \end{lstlisting}

  We can note that the GPU output is identical to that of our CPU, and Wolfram Alpah, until the 6th decimal.

\end{itemize}

\end{document}